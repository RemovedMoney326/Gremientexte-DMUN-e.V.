\documentclass[a4paper,11pt]{article}

\usepackage{ngerman}
\usepackage{soul}
\usepackage{mathtools}
\usepackage{amssymb,amsmath,amsfonts}
\usepackage[utf8]{inputenc}
\usepackage{graphicx}
\usepackage{geometry}
\usepackage{float}
\usepackage[german=quotes]{csquotes}
\usepackage{hyperref}
\usepackage{fancyhdr}
\usepackage{gensymb}
\usepackage{units}
\usepackage{hhline}
\usepackage{color}
\usepackage[export]{adjustbox}
\usepackage[nottoc,numbib]{tocbibind}
\usepackage[super,comma]{natbib}
\usepackage{titling}

\geometry{a4paper, left=30mm, right=30mm, top=30mm, bottom=30mm}
\definecolor{pantone294}{cmyk}{1,0.6,0,0.2}

\title{Richtlinien zum Umgang mit Technologien künstlicher Intelligenz}
\author{Vorsitz Generalversammlung \\ \\ Santiago Rodriguez}
\date{20.8.2019}
\pagestyle{fancy}
\lfoot{MUN-SH 2020}
\rfoot{Richtlinien zum Umgang mit K.I}

\begin{document}
	\newgeometry{left=14mm, right=13.5mm, top=13.5mm, bottom=30mm}
	\begin{titlepage}
		\thispagestyle{empty}
		\begin{figure}
			\includegraphics[width=31.5mm,right]{./munshlogo.png}
		\end{figure}
		\vspace*{-43mm}\hspace{-6mm}\textbf{\textcolor{pantone294}{\large{DMUN}}}\\\\\\\\\\
		\textcolor{pantone294}{MUN-SH 2020}\\
		\vspace{30mm}
		\begin{center}
			\textcolor{pantone294}{\huge{MUN-SH 2020}}\\\vspace*{7mm}
			\textcolor{pantone294}{\huge{\textbf{\thetitle}}}\\\vspace*{10mm}
			\textcolor{pantone294}{\theauthor}\\\vspace*{10mm}
			\textcolor{pantone294}{\thedate}\\\vspace*{20mm}
		\end{center}
	\end{titlepage}
	\makeatother
	\restoregeometry
	\newpage
	
	\tableofcontents
\vspace{2cm}
	
	
	\section{Einleitung}
Digitale Hilfsmittel prägen das Alltagsleben von Menschen seit Ende des letzten Jahrhunderts. Von der Mondlandung bis zu \textbf{Globalisierung} der Kommunikation haben \textbf{IT-Systeme} eine Vielzahl an Entwicklungen in Bereichen wie der Politik, Wirtschaft, und den Naturwissenschaften ermöglicht, die ohne sie undenkbar gewesen wären. Seit Beginn dieses Jahrzehnts haben sich außerdem immer mehr IT-Systeme entwickelt, die in der Lage sind, komplexe Probleme mithilfe flexibler Programme zu prozessieren. Diese autonomen Systeme bezeichnet man im allgemeinen Sprachgebrauch als künstliche Intelligenz, abgekürzt KI. Sie sind in der Lage, Problemstellungen, die bisher exklusiv von Menschen angegangen werden konnten, selbstständig zu lösen. 
Mit dieser Fähigkeit geht jedoch eine Vielzahl an Problemen einher. Trotz der Fähigkeit solcher Programme, komplexe Aufgaben anzugehen, kommt es vor, dass sie Entscheidungskriterien, die für ein Problem auf subtilerer Ebene relevant sind, ignorieren oder relevante Fähigkeiten nicht besitzen. Darunter können beispielsweise die spontane Entscheidungsfähigkeit und ethisch-moralische Bewertungen bei Extremsituationen wie einem Autounfall fallen. Darüber hinaus kann ein IT-System nicht die juristischen Konsequenzen seiner Handlungen tragen. \\
Neue ethische und juristische Richtlinien sind somit notwendig, um die Entwicklung und Anwendung von KI basierten Systemen zu regulieren sowie deren Missbrauch zu vermeiden. Einen international anerkannten \textbf{Richtlinienkatalog} zu erstellen wäre ein wichtiger erster Schritt beim Umgang mit solchen Technologien. Die immer rascher voranschreitende Entwicklung von KI Technologien könnte sich dann an diesen Richtlinien orientieren. Sie könnten ebenfalls als Hinweise für die nationale Gesetzgebung der Mitglieder der \textbf{internationalen Staatengemeinschaft} dienen und somit den Aufbau rechtlicher Rahmen für KI beschleunigen. 

  \newpage

	\section{Hintergrund und Grundsätzliches}
Unter dem Begriff \textit{Künstliche Intelligenz} versteht man im Allgemeinen die Eigenschaft eines IT-Systems, bei der Durchführung eines Programms menschenähnliche, intelligente Verhaltensweisen zu zeigen. Darunter fällt unter anderem die Fähigkeit, dass ein KI-fähiges IT-System durch wiederholtes ansammeln und verarbeiten von Daten selbstständig neue Erkenntnisse erlernt, die dann für die Erfüllung einer Aufgabe genutzt werden. Diese Erfahrungen werden beispielsweise durch eine wiederholte Durchführung des Programms und einer Bewertung der Ergebnisse gewonnen. Anhand dieser Bewertungen kann das System dann die Durchführung des Programms verbessern. Ein solches Verfahren bezeichnet man als \textbf{\textit{maschinelles Lernen}}. So kann ein KI System beispielsweise durch wiederholtes Hören einer menschlichen Stimme besser darin werden, gesprochene Wörter zu erkennen. \\ \\ Dies setzt aber eine hohe \textbf{Rechenkapazität} voraus. Da eine solche Rechenkapazität bis zum aktuellen Jahrzehnt außerhalb von großen, kostspieligen Rechenzentren nicht verfügbar war, ist die Entwicklung von KI bis vor Kurzem ein recht spezielles  Forschungsgebiet gewesen. Nun, da die notwendige Rechenkapazität zur Verfügung steht und die weltweite Vernetzung es ermöglicht, von fast überall mit einem mobilen Gerät auf solche hochleistungsfähige Rechenzentren zuzugreifen, ist die weltweite Forschung und Entwicklung von KI basierten Systemen und Programme für eine Vielzahl von Entwickler*innen möglich geworden. Insbesondere das ausgesprochen breite Spektrum an Anwendungen solcher Technologien verlockt viele Privatunternehmen und globale Techfirmen, in diesem Feld zu investieren und Forschungsprojekte sowie spezialisierte Entwickler*innen zu finanzieren. So sind zurzeit alle Mitglieder der sogenannten \textit{Big 4} Techfirmen (Apple, Amazon, Facebook und Google) an der Entwicklung von Anwendungen von KI basierten IT-Systemen beteiligt. Unter den möglichen Anwendungen findet man unter Anderem \textbf{selbstfahrende Fahrzeuge}, automatische Bild- und Videobearbeitungs-, sowie \textbf{Gesichtserkennungsprogramme} und \textbf{autonome Waffensysteme}. Diese breite Anwendbarkeit hat in den letzten Jahren zu einer Vielzahl neuer KI basierter Programme geführt, mit der andere Bereiche der menschlichen Entwicklung wie Rechtsprechung oder Ethik kaum mithalten konnten. Die moralische Vertretbarkeit autonomer Waffensysteme, die zukünftig in der Lage sein könnten, ohne jegliche menschliche Kontrolle Zielobjekte zu töten, ist beispielsweise oft in Frage gestellt worden. \\ \\ Die nationale Sicherheit von Staaten könnte aufgrund der immer weiter voranschreitenden Digitalisierung sensibler Bereiche wie des Militärs oder des Gesundheitswesens ebenfalls durch künstliche Intelligenzen gefährdet werden. Diese finden in modernen Arten von Cyber-Angriffen wie \textbf{DDoS (Distributed-Denial-of-Service, dt.: Verteilte-Verweigerung des Dienstes) Attacken} eine breite Verwendung und ermöglichen es, große Angriffe mit nur einer kleinen Anzahl an Personen durchzuführen. 
Auch in der Wirtschaft sind KI basierte Programme ein möglicher Grund zur Sorge, da sie im Vergleich zu konventionellen Programmen in der Lage sind, in Aufgabenbereiche, bei denen bisher exklusiv menschliche Arbeitskräfte verwendet werden könnten, einzuschreiten. Die maschinelle Übernahme vollständig neuer Kompetenzen, die bisher im allgemeinem Recht und in der Verwaltung von Unternehmen nicht für IT-Systeme vorgesehen gewesen sind, stellt ebenfalls ein Grund zur Sorge dar. Weitreichende Verluste von Arbeitsplätzen in Bereichen wie Logistik oder Fertigung von Produkten werden in den nächsten Jahren nach Ergebnissen vieler Studien erwartet, was zusätzlich auf eine Frage der Anstellungssicherheit in der Welt mit KI führt.   

  \newpage		

    \section{Aktuelles}
   Momentan befindet sich die Entwicklung von KI basierten Technologien auf einem immer schnelleren Weg dazu, in das Alltagsleben von Menschen weltweit vorzudringen. Einige Staaten haben bereits begonnen mit Gesichtserkennungssoftware zur automatischen Erkennung von Kriminellen zu experimentieren und möchten die Technologie in Zukunft für die Überwachung einer großen Anzahl an Individuen verwenden können. Andere wenden künstliche Intelligenz seit Jahren bei automatischen Drohnen im militärischen Bereich an, so wie bei dem vor kurzem entwickelten Boeing MQ-25 Stingray zur Luftbetankung bemannter Flugzeuge. Voraussichtlich werden in der nahen Zukunft Drohnen mit offensiven Anschlagskapazitäten hergestellt werden können. Auch im Netz wird derzeit künstliche Intelligenz zur Verbreitung von Falschinformationen angewendet. Mit den umgangssprachlich als \textit{Web-Brigaden} bezeichneten Programmen, die auf verschiedene Weisen im Netz falsche Gerüchte und Nachrichten verbreiten, können Stimmungs- und Meinungsverhältnisse auf Onlineforen und Sozialen Netzwerken zugunsten nationaler Interessen beeinflusst werden. Auch neue KI basierte Softwareanwendungen wie DeepFake oder FaceApp, die in der Lage sind Videos oder Bilder automatisch zu bearbeiten, um falsche Reden von Politiker*innen oder künstlich gealterte Fotografien von Menschen herzustellen, können auf national gesponsorte Softwareentwickler zurückgeführt werden. Unter den Staaten, die am meisten an der Entwicklung solcher Technologien beteiligt sind, befinden sich die Vereinigten Staaten, die russische Föderation sowie die Volksrepublik China. \\ Zusätzlich wurde im Bereich der Geschlechtergerechtigkeit im Sommer 2019 eine KI basierte App namens DeepNude diskutiert, die in der Lage war, automatisch Bilder von gekleideten Frauen auf einer solchen Weise zu verarbeiten, dass diese realistisch ohne jegliche Kleidung angezeigt werden könnten. Einen rechtlichen Präzedenzfall, um eine formale Anklage gegen den Entwickler erheben zu können und falsche Fotos aus dem Netz zu entfernen, gab es nicht. Da es bei künstlicher Intelligenz noch sehr wenige Präzedenzfälle gibt und ein umfangreicher rechtlicher Rahmen nicht existiert, sind solche Gesetzeslücken häufig vorzufinden. Eine ähnliche Problemstellung entstand im Jahre 2018, als in den Vereinigten Staaten zum ersten Mal ein Verkehrsunfall stattfand, bei dem ein selbstfahrendes Fahrzeug der Firma Uber eine Fußgängerin ums Leben brachte. Die Untersuchung brauchte aufgrund der völlig neuen Natur des autonomen Wagens deutlich mehr Zeit und stellte das damalige Untersuchungsgericht vor eine Vielzahl an neuen Fragen, die bisher nicht bei gewöhnlichen Verkehrsunfällen vorgekommen waren. \\ \\ Jenseits dieser rechtlichen und ethischen Umstände ist es jedoch in Gebieten wie der Forschung und den Naturwissenschaften dank KI basierter Anwendungen zur Lösung von bisher äußerst komplizierten oder sogar analytisch unlösbaren Problemstellungen gekommen. So ist in der Biochemie seit kurzem ein KI basiertes Programm namens AlphaFold dazu in der Lage, die dreidimensionale Struktur von Proteinen aus Aminosäureketten in deutlich kürzer Zeit richtig vorherzusagen, als menschliche Forschungsgruppen dies können. Die numerische Lösung mathematischer Probleme kann ebenfalls durch die Anwendung von KI in spezialisierten Programmanwendungen wie Mathematica's \textit{Neural Networks} deutlich erleichtert werden. Und auch jenseits hiervon gibt es eine breite Anzahl von Anwendungen, die das allgemeine Wohl der Gesellschaft weltweit fordern könnten.

  \newpage
    
    \section{Probleme und Lösungsansätze}
Beim Umgang mit diesen neuen KI basierten Technologien stellt sich vor allem die Frage, nach welchen ethischen und rechtlichen Maßstäben sich die Regulierung und Entwicklung selbiger richten muss. \\ Die Europäische Union hat hierzu bereits eine Richtlinie ethischer Maßstäbe für vertrauenswürdige künstliche Intelligenzen veröffentlicht, die unter anderem vorsieht, dass sich diese an bereits bestehende Gesetzgebungen halten und grundlegende ethische Prinzipien berücksichtigen müssen. Die EU gibt aber unter diesen Maßstäben beispielsweise keine Hinweise dafür, wer die rechtliche Verantwortung bei Missbrauch künstlicher Intelligenzen trägt. \\ Konkretere Richtlinien zum zukünftigen Umgang mit künstlicher Intelligenz wurden vom Verband der Internetwirtschaft (ECO) aufgestellt, der eine weit gefächerte Anzahl an neuen Maßnahmen und Strukturen vorschlug, um die sachgemäße Nutzung der Technologie in Zukunft gewährleisten zu können. So wurde in diesen Richtlinien unter anderem vorgeschlagen, dass die Nutzung Künstlicher Intelligenz im Alltag als Wirtschafts- und Standortfaktor anerkannt wird. Denn ihre streng wissenschaftlichen Anwendungen haben KI-fähige Systeme mittlerweile hinter sich gelassen. Bereits seit einigen Jahren spielen sie in Sektoren wie die Wirtschaft oder die Gesundheit eine Rolle. Sollten diese Anwendungen die ökonomische Attraktivität des jeweiligen Anwendungsortes beeinflussen, können Künstliche Intelligenzen als ein Standortfaktor betrachtet werden. \\ Diese Richtlinien haben jedoch noch keine breite Anerkennung erlangt. Noch immer fehlt ein Bekenntnis der internationalen Staatengemeinschaft zu einem konkreten Katalog an Richtlinien. Diesen ersten, international anerkannten Richtlinienkatalog zu erstellen wäre somit ein solider Anfang, um den verantwortungsvollen Umgang mit künstlicher Intelligenz in Zukunft gewährleisten zu können. Unter diesen Richtlinien könnte man Punkte der Richtlinien vom Verband der Internetwirtschaft einarbeiten, sowie auch neue, innovative Regelungen und Grenzsetzungen die von der Staatengemeinschaft als notwendig erachtet werden. Wenn das Gremium der Ansicht ist, das diese Fragestellungen konkreter ausgeführt werden sollten, könnte es unter Umständen angemessen sein, diese an eine bestehende Expertenkommission für die Findung von Regularien in einzelnen Anwendungsgebiete (z.B. Waffensysteme, Gesundheitssektor) zu überweisen. \\ Wichtig ist, dass sich die Generalversammlung über die nicht-\textbf{verbindliche} Natur solcher Richtlinien bewusst ist. Damit eine Umsetzung garantiert werden kann, sollten die Positionen von Ländern, die zurzeit im Feld der Forschung künstlicher Intelligenzen am meisten aktiv sind, bei den Verhandlungen besonders beachtet werden. \\ \\ Bei den Debatten sollte vor allem die Frage im Vordergrund stehen, welche Grenzen und Ziele beim Umgang mit künstlicher Intelligenz allgemein aufgestellt und beachtet werden sollten. Konkrete, fachspezifische Fragen wie der richtige Umgang mit solchen Technologien bei unterschiedlichen, militärischen Anwendungen sollten bei Bedarf an spezialisierte Gremien wie den Hauptausschuss 1 übergeben werden. Auch eine Debatte darüber, wie genau künstliche Intelligenzen definiert werden können oder was damit in Zukunft möglich sein wird, überschreitet die Grenzen dieses Tagesordnungspunktes und könnte eventuell von der Kommission für Wirtschaft und Technologie aufgenommen werden. Die Frage für die Generalversammlung ist im wesentlichen nicht, wozu künstliche Intelligenz in der Lage ist oder sein kann, sondern zu was sie aus internationaler Sicht in der Lage sein soll und zu was nicht.

  \newpage
      
    \section{Punkte zur Diskussion}

-Sollte es eine Grenzsetzung bei der Forschung und Entwicklung künstlicher Intelligenzen geben? Wenn ja, welche? Sollte es bspw. eine Regulation darüber geben, welche Institutionen zuverlässig genug sind, um KI entwickeln zu können? \\ \\ -Wer sollte bei Missbrauch künstlicher Intelligenzen die juristische Verantwortung tragen? Ist das eher fallspezifisch und wenn ja, unter welchem rechtlichen Rahmen sollte man bei solchen Fällen den Missbrauch untersuchen? \\ \\ -Inwiefern sollten KI basierte Systeme unabhängig vom Benutzer agieren können? Wann sollte es notwendig sein, für die Durchführung der Aufgabe eines KI basierten Programms die manuelle Bestätigung durch eine verantwortlichen Person zu fordern? \\ \\ -Welche Punkte aus bereits bestehenden Richtlinien wie denen der Europäischen Union oder des Verbandes der Internetwirtschaft können aus internationaler Sicht übernommen werden und in welcher Form? \\ \\ -Sollten Expertenkommissionen oder die Hauptausschüsse der Generalversammlung damit beauftragt werden, zu einem bestimmten Anwendungsbereich künstlicher Intelligenzen konkretere Richtlinien zu finden? Falls ja, welche Bereiche sollten konkreter in Betracht gezogen werden und von wem? \\ \\ -Welche zukünftigen Möglichkeiten durch KI, insbesondere im Bereich der Wirtschaft oder der wissenschaftlichen Forschung, sollten gefördert und welche gehemmt werden? Sollten künstliche Intelligenzen in der Lage sein, alle Tätigkeiten zu übernehmen zu deren Ausführung sie fähig sind, oder sollten sensible Aufgaben, die bspw. so wie die eines Flugzeugpiloten eine Verantwortung über das Leben einer großen Anzahl von Menschen mit sich bringen, eher menschlichen Arbeitskräften überlassen werden?


\vspace{1cm}

   \section{Lexikon}

Autonome Waffensysteme: Drohnen und andere selbstgesteuerte Gefährte die in der Lage sind, mithilfe montierter Sensoren und Waffen selbstständig Ziele aufzusuchen und zu neutralisieren. \\ \\ DDoS Attacke: Angriff bei dem ein Anbieter von Dienstleistungen im Netz von einer massiven Anzahl an Anfragen überfordert und somit neutralisiert wird. \\ \\ Expertenkommission: In diesem Falle wären es Neben- und Unterorgane der Vereinten Nationen so wie bspw. die Hauptausschüsse 1-6 der Generalversammlung, die Kommission für Wirtschaft und Technologie oder Institute der Vereinten Nationen wie das UNICRI (United Nations Interregional Crime and Justice Research Institute). \\ \\ Gesichtserkennungsprogramme: Programme die in der Lage sind, durch Vergleich mit einer Datenbank von Bildern und Namen eine Person in einer Bild- oder Videoaufnahme zu identifizieren. \\ \\ Globalisierung: Weltweite Verflechtung in Bereichen wie die Wirtschaft, Politik, Kultur u. a. \\ \\ Internationale Staatengemeinschaft: Gemeinschaft der 193 Mitgliedsstaaten der Vereinten Nationen. \\ \\ IT-Systeme: Abgekürzte Form aus dem Englischen \textit{Information Technology Systems}; diese Bezeichnung stellt eine allgemeine Form dar, um sich auf Computer, Mobiltelefone, Fernseher und andere digitale Speichermedien mit der Fähigkeit, Daten zu bearbeiten, zu beziehen. \\ \\ Maschinelles Lernen: Verfahren bei dem ein IT-System durch wiederholte Durchführung eines Programms neue Daten sammelt um bei zukünftigen Durchführungen ein besseres Ergebnis zu liefern. \\ \\ Rechenkapazität: Fähigkeit eines Computers oder IT-Systems, ein Programm oder eine Berechnung innerhalb eines bestimmten Zeitintervalls durchzuführen. \\ \\ Richtlinienkatalog: Gruppierung von Regeln, an denen sich ein bestimmtes Bereich menschlicher Aktivitäten zu richten hat. \\ \\ Selbstfahrende Fahrzeuge: Fahrzeuge die dazu fähig sind, gemäß der etablierten Verkehrsregeln ohne jegliche menschliche Kontrolle auf dem Verkehrsnetz zu fahren. \\ \\ Verbindlichkeit: In diesem Kontext \textbf{völkerrechtlicher} Fachbegriff bei einer Resolution, nach dem die Verpflichtung eines Staates, sich an den Inhalt einer Resolution zu halten, verdeutlicht wird. Nur der Sicherheitsrat kann jedoch völkerrechtlich verbindliche Resolutionen verabschieden. \\ \\ Völkerrecht:  International verbindliches, besonders zwischenstaatliches Recht 

  \section{Weiterführende Links und wichtige Dokumente}
  
\href{https://www.itu.int/en/ITU-T/AI/Pages/201706-default.aspx}{-AI for Good Global Summit der ITU [English]} \\ \\ \href{http://www.unicri.it/in_focus/on/UNICRI_Centre_Artificial_Robotics}{-UNICRI Centre for Artificial Intelligence and Robotics [English]} \\ \\ \href{https://ec.europa.eu/digital-single-market/en/news/ethics-guidelines-trustworthy-ai}{-Ethik Leitlinien der EU [Englisch und Deutsch]} \\ \\ \href{https://www.eco.de/wp-content/uploads/dlm_uploads/2018/09/20180918_eco_LTL-K\%C3\%BCnstliche-Intelligenz.pdf}{-Richtlinien zum Umgang mit künstlicher Intelligenz des Verbandes der Internetwirtschaft [Deutsch]} \\ \\ \href{https://www.un.org/sustainabledevelopment/blog/2017/10/looking-to-future-un-to-consider-how-artificial-intelligence-could-help-achieve-economic-growth-and-reduce-inequalities/}{-Konferenzwebsite des ECOSOC „The future of everything“ [English]} \\ \\ \href{http://www.zeit.de/digital/internet/2017-10/kuenstliche-intelligenz-deepmind-back-box-regulierung}{-Artikel auf Zeit Online zur Kontrolle von künstlicher Intelligenz [Deutsch]} \\ \\ \href{https://www.iotforall.com/impact-of-artificial-intelligence-job-losses}{-Studie zum Verlust von Arbeitsplätzen aufgrunde künstlicher Intelligenzen [Deutsch]} \\ \\ \href{https://www.nature.com/articles/d41586-019-01357-6}{-Artikel zum Beitrag AlphaFolds im Forschungsbereich von Proteinstrukturen [Englisch]} \\ \\ \href{https://www.theguardian.com/technology/ng-interactive/2019/jun/22/the-rise-of-the-deepfake-and-the-threat-to-democracy}{-Bericht von The Guardian zum Thema DeepFakes und ihre Bedeutung für die moderne Politik [Englisch]} \\ \\ \href{https://www.washingtonpost.com/business/2019/06/28/the-world-is-not-yet-ready-deepnude-creator-kills-app-that-uses-ai-fake-naked-images-women/}{-Artikel von The Washington Post zur Anwendung von DeepFakes in der mobilen App DeepNude, sowie dessen Entfernung aus dem Netz [Englisch]} \\ \\ \href{https://www.spiegel.de/wissenschaft/technik/autonome-waffen-ausser-kontrolle-a-1253320.html}{-Spiegel Online Artikel zur Zukunft autonomer Waffensysteme [Deutsch]} \\ \\ \href{https://www.zeit.de/digital/internet/2019-04/kriegssysteme-autonome-waffen-digitalisierung-wird-das-was-digitalpodcast}{-Ausführlicher Podcast der Zeit zum selbigen Thema autonomer Waffensysteme [Deutsch]}
 
\end{document}
