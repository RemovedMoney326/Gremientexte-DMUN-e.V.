\documentclass[a4paper,11pt]{article}

\usepackage{ngerman}
\usepackage{soul}
\usepackage{mathtools}
\usepackage{amssymb,amsmath,amsfonts}
\usepackage[utf8]{inputenc}
\usepackage{graphicx}
\usepackage{geometry}
\usepackage{float}
\usepackage[german=quotes]{csquotes}
\usepackage{hyperref}
\usepackage{fancyhdr}
\usepackage{gensymb}
\usepackage{units}
\usepackage{hhline}
\usepackage{color}
\usepackage[export]{adjustbox}
\usepackage[nottoc,numbib]{tocbibind}
\usepackage[super,comma]{natbib}
\usepackage{titling}

\geometry{a4paper, left=30mm, right=30mm, top=30mm, bottom=30mm}
\definecolor{pantone294}{cmyk}{1,0.6,0,0.2}

\title{Überprüfung der Sanktionen und Embargos gegenüber der Islamischen Republik Iran}
\author{Vorsitz Sicherheitsrat \\ \\ Santiago Rodriguez}
\date{20.9.2018}
\pagestyle{fancy}
\lfoot{MUN-SH 2019}
\rfoot{Sanktionen und Embargos gegenüber dem Iran}

\begin{document}
	\newgeometry{left=14mm, right=13.5mm, top=13.5mm, bottom=30mm}
	\begin{titlepage}
		\thispagestyle{empty}
		\begin{figure}
			\includegraphics[width=31.5mm,right]{./munshlogo.png}
		\end{figure}
		\vspace*{-43mm}\hspace{-6mm}\textbf{\textcolor{pantone294}{\large{DMUN}}}\\\\\\\\\\
		\textcolor{pantone294}{MUN-SH 2019}\\
		\vspace{30mm}
		\begin{center}
			\textcolor{pantone294}{\huge{MUN-SH 2019}}\\\vspace*{7mm}
			\textcolor{pantone294}{\huge{\textbf{\thetitle}}}\\\vspace*{10mm}
			\textcolor{pantone294}{\theauthor}\\\vspace*{10mm}
			\textcolor{pantone294}{\thedate}\\\vspace*{20mm}
		\end{center}
	\end{titlepage}
	\makeatother
	\restoregeometry
	\newpage
	
	\tableofcontents
\vspace{2cm}
	
	
	\section{Einleitung}
Die globale Nichtverbreitung und Abrüstung von Nuklearwaffen, sowie die freie Bereitstellung von Nuklearenergie für zivile Zwecke, sind Aufgaben im Rahmen des Atomwaffensperrvertrags, mit denen sich sowohl die Vereinten Nationen als auch die Internationale Atomenergie Organisation (IAEO) seit 1970 beschäftigen. Neben biologischen und chemischen Waffen zählen Nuklearwaffen zu den Massenvernichtungswaffen. Deswegen können Staaten, die nicht als offizielle Atommächte gelten, aber trotzdem versuchen, Atomwaffen zu entwickeln, eine weitreichende Gefahr für den Weltfrieden und die Stabilität der internationalen Ordnung darstellen, weshalb sich der UN-Sicherheitsrat solcher Fälle annimmt. Als einziges Organ der Vereinten Nationen mit der Fähigkeit, Resolutionen mit völkerrechtlich bindender Wirkung zu verabschieden, hat er gemäß der Charta der Vereinten Nationen die Verantwortung, für die Wahrung des Weltfriedens und der internationalen Sicherheit zu sorgen.
\\ \\
In Form des Atomprogramms der Islamischen Republik Iran liegt dem Sicherheitsrat seit 2006 ein Fall vor, bei dem der widerrechtliche Erwerb von Kernwaffen vermutet wird. Aufgrund mehrerer Kommunikations- und Transparenzprobleme in der Zusammenarbeit mit der IAEO sowie der Ausführung bestimmter Einzelprogramme liegt der Verdacht nahe, dass das zivile Atomprogramm Irans als Tarnung für die geheime Entwicklung von Kernwaffen benutzt wird. Dies hat in den folgenden Jahren zu gemeinsamen Sanktionen und Embargos durch die USA, die EU und die UN geführt. Der resultierende wirtschaftliche Druck auf den Iran hat 2015 zu dem Abschluss des Gemeinsamen Umfassenden Aktionsplans (Joint Comprehensive Plan of Action, JCPOA) geführt, in dem eine scharfe Einschränkung und Kontrolle der nuklearen Bestrebungen des Irans vereinbart wurden. Nachdem bestätigt wurde, dass der Iran das Abkommen einhält, wurden noch im selben Jahr die ersten Sanktionen und Embargos aufgehoben und das Thema ausgesetzt. Mit dem einseitigen Austritt der USA im Mai 2018 und der Wiedereinführung amerikanischer Sanktionen gegen den Iran sind aber nun die rechtlichen Umstände des JCPOAs zusammen mit dessen Zukunft in Frage gestellt worden und der Umgang mit dem Atomprogramm des Irans steht somit wieder auf der Tagesordnung des Sicherheitsrats.



	\section{Hintergrund und Grundsätzliches}
"Dieser Vertrag ist nicht so auszulegen, als werde dadurch das unveräußerliche Recht aller Vertragsparteien beeinträchtigt, unter Wahrung der Gleichbehandlung und in Übereinstimmung mit den Artikeln I und II die Erforschung, Erzeugung und Verwendung der Kernenergie für friedliche Zwecke zu entwickeln.“ - Vertrag über die Nichtverbreitung von Kernwaffen, Artikel IV
\\ \\
1968 initiierten die Vereinigte Staaten, die Sowjetunion, das Vereinigte Königreich, Frankreich und China als die fünf offiziellen Atommächte den ''Vertrag über die Nichtverbreitung von Kernwaffen'' (NVV) . Dieser Vertrag war im Rahmen des Kalten Krieges entstanden, um die globale Gefahr eines unmittelbaren Atomkriegs zu verringern. Mit Unterzeichnung des Vertrages verpflichteten sich Nicht-Atommächte wie der Iran unter Artikeln 1 und 2 dazu, nukleare Technologien nicht für militärische Zwecke zu verwenden und der IAEO Informationen über jegliche nuklearen Aktivitäten im Lande selbstständig zu melden. Der Vertrag sieht aber unter Artikel 4 auch vor, dass allen Unterzeichnerstaaten die friedliche und zivile Nutzung von Kernenergie zusteht. Der Vertrag wurde weltweit von 191 Staaten unterzeichnet, zu denen auch der Iran gehört, und dient seitdem als allgemeine Richtlinie zur Entwicklung nuklearer Technologien auf ziviler und militärischer Ebene.
\\ \\
Der Konflikt um das iranische Atomprogramm nahm vor diesem Hintergrund seinen Anfang im Jahre 2002, als aufgrund einer inländischen Quelle die Existenz mehrerer Einrichtungen zur Anreicherung von Uran auf iranischem Boden bestätigt wurde. Der Iran hatte diese Einrichtungen der IAEO nicht gemeldet. Die Anreicherung von Uran ist eine künstliche Weiterverarbeitung von natürlichem Uran, um dessen Nutzung in Kernkraftwerken zu ermöglichen. Bei besonders starker Anreicherung dient es aber auch zur Herstellung von Kernwaffen, weshalb die Geheimhaltung dieser Einrichtungen als ein starkes Indiz für ein nukleares Sprengkopfprogramm angesehen wurde. Noch in den folgenden Monaten stellte der Iran sämtliche Informationen über sein Atomprogramm der IAEO zur Verfügung und unterzeichnete 2003 ein Zusatzprotokoll des NVV, das unter anderen Sicherheitsmaßnahmen auch unangemeldete Inspektionen durch die IAEO ermöglichte.
\\ \\
Die Ratifikation der Unterzeichnung des Zusatzprotokolls durch Iran blieb aber aus und im folgenden Jahr bescheinigte die IAEO dem Iran nach mehreren angemeldeten Inspektionen ungenügende Kooperationsbereitschaft. Dies führte zu einer weiteren Eskalation der Situation. Währenddessen bestand die iranische Regierung aber auf der rein zivilen Natur seines Atomprogramms und wies den Verdacht der IAEO als unbegründet zurück. Nachdem der Iran im Jahr 2006 die Anreicherung von Uran wieder aufnahm und der IAEO die Beendigung seiner freiwilligen Kooperation mitteilte, erhoben unter anderem die USA und die EU Sanktionen gegen den Iran und der Sicherheitsrat beschäftigte sich zum ersten Mal mit dem Fall.
\\ \\
Der Sicherheitsrat verhängte noch im Dezember desselben Jahres mit Resolution 1737 internationale Sanktionen gegen den Iran und forderte die sofortige Einstellung von dessen Urananreicherungsprogramm. Der Iran reagierte mit einer Absage auf die Forderung, sein Uranprogramm einzustellen. Deswegen verhängte der Sicherheitsrat in den folgenden Jahren weitere Sanktionen und Embargos gegen die islamische Republik, während gleichzeitig mehrere Versuche eines gemeinsamen Abkommens scheiterten.
\\ \\
2015 kam dann nach mehreren Jahren diplomatischer Anstrengungen eine Einigung zustande, die unter dem Namen Joint Comprehensive Plan of Action (JCPOA) von den P5+1 (ständige Mitglieder des UN-Sicherheitsrates und Deutschland), der EU und vom Iran unterzeichnet wurde. Hierbei handelte es sich formell um eine nicht-verbindliche Einigung der Unterzeichnerstaaten, bei der sich die islamische Republik Iran unter anderem darauf verpflichtete, seinen Bestand an angereichertem Uran um etwa 95\% zu reduzieren, die für die Anreicherung nötigen Zentrifugen von etwa 19.000 auf 6.000 zu vermindern und den Schwerwasserreaktor in Arak, der zur Produktion von waffenfähigem Plutonium geeignet war, außer Betrieb zu nehmen. Im Gegenzug billigten die anderen Vertragsparteien die eingeschränkte Weiterführung des iranischen Atomprogramms und erklärten sich zu einer schrittweisen Aufhebung der Sanktionen bereit. Die Resolution 2231 des Sicherheitsrates befürwortete den JCPOA und sah neben einem Inspektionsprozess auch den Abbau der Sanktionen gegen den Iran vor. Die Resolution sieht auch vor, dass bei Nichteinhaltung des JCPOA durch den Iran die Sanktionen unverzüglich erneut verhängt werden können. Die IAEO bestätigte aber im Folgenden, dass die iranische Regierung sich an die Vereinbarung hielt. Der Konflikt galt für’s Erste als gelöst und die ersten Sanktionen wurden im Januar 2016 aufgehoben.


    \section{Aktuelles}
  Bedeutung erlangte die Situation erneut im Mai 2018, als die derzeitige US-Regierung unter Präsident Donald Trump den Austritt der USA vom JCPOA ankündigte. Der amerikanische Präsident hatte sich schon während seiner Wahlkampagne höchst kritisch gegenüber dem Abkommen seines Vorgängers geäußert und hatte die iranische Regierung mehrmals beschuldigt, sich nicht an das Abkommen gehalten zu haben. Er kritisierte auch weitere Punkte am JCPOA, wie z.B. die fehlende Erwähnung oder Beschränkung des iranischen Raketenprogramms und die Unterstützung von Gruppen wie Hamas oder Hisbollah durch das iranische Regime, da diese von der amerikanischen Regierung als Terrororganisationen betrachtet werden. Nach dem Austritt der USA verkündeten die restlichen Unterzeichnerstaaten, weiter an dem Abkommen festhalten zu wollen, und baten den Iran, sich trotz des amerikanischen Austritts aus dem Abkommen weiterhin an die Bedingungen des JCPOAs zu halten.
\\ \\
Mit der Wiederaufnahme sogenannter amerikanischer Sekundärsanktionen im August 2018 wurden allerdings effektiv nicht nur amerikanische Unternehmen von dem Handel mit dem Iran abgehalten, sondern auch ausländische. Im Rahmen dieser Sanktionen erhalten Unternehmen, die mit dem Iran handeln, keinen Zugang mehr zum amerikanischen Markt. Da der amerikanische Markt einer der wichtigsten ist, erstrecken sich so auf indirekte Weise die amerikanischen Sanktionen auch auf u.a. europäische Unternehmen.
\\ \\
Die amerikanische Regierung hat mit einer weiteren Verschärfung der Sanktionen im August gedroht und fordert die Einstellung der Urananreicherung für Wiederaufnahme der Verhandlungen. Der Iran hat aber mit scharfer Kritik reagiert und den Vereinigten Staaten Vertragsbruch vorgeworfen. Laut Auffassung des Irans hat die IAEO entgegen der Aussagen des amerikanischen Präsidenten regelmäßig die Einhaltung des Abkommens durch den Iran bestätigt, weshalb der Austritt der Vereinigten Staaten aus dem Abkommen und die einseitige Wiederaufnahme von Sanktionen nicht gerechtfertigt sei. Trotzdem hat die iranische Regierung angekündigt, sich weiterhin an das Abkommen zu halten, obwohl die langfristige Zukunft des JCPOAs nach Meinung vieler Experten nun nicht mehr sicher sei.

    
    \section{Probleme und Lösungsansätze}
Ein zentrales Problem der derzeitigen Situation ist die Frage nach der Rechtmäßigkeit der einseitigen Verhängung von Sanktionen durch die amerikanische Regierung. Denn aufgrund der vorhin erwähnten Sekundärsanktionen werden derzeitig nicht nur amerikanische Unternehmen bestraft, sondern auch ausländische Unternehmen, die mit der islamischen Republik innerhalb ihrer Landesgesetze rechtmäßig handeln. Die amerikanische Regierung zwingt somit ihre außenpolitische Stellung gegenüber dem Iran anderen Ländern auf, die mit der USA Handel betreiben wollen. Das schränkt auch die Fähigkeit der anderen Vertragsparteien ein, sich an den JCPOA wie vereinbart zu halten. Außerdem sind die rechtlichen Linien des Austritts der USA aus dem JCPOA ungenau formuliert und derzeit umstritten. Der JCPOA hat selbst als nicht verbindliche Einigung der Vertragsparteien keinen zwingenden Charakter und ermöglicht jederzeit den Austritt eines Unterzeichners. Doch die Befürwortung des Sicherheitsrates in der Resolution 2231 fügte dem JCPOA eine völkerrechtlich verbindliche Dimension hinzu, insbesondere im Hinblick auf den graduellen Abbau der Sanktionen. Aber selbst im Fall der Resolution 2231 wird weiterhin die Verbindlichkeit diskutiert. Die Bindungswirkung einer Resolution des Sicherheitsrates hängt von der Sprache und den verwendeten Operatoren ab. Entsprechend gibt es mehrere Stimmen, die auf einen verbindlichen Charakter der Aussagen zu den Sanktionen bestehen. Gleichzeitig gibt es auch Gegenstimmen, die die Resolution bloß als eine Empfehlung im Rahmen des freiwillig einzuhaltenden JCPOAs betrachten. Die rechtlichen Umstände um den Austritt der USA aus dem JCPOA und die einseitige Wiederaufnahme von Primär- und Sekundärsanktionen durch die USA sollten somit besprochen und geklärt werden, sowie die langfristige Zukunft des JCPOAs selbst und gegebenenfalls mögliche Ansätze zur Neuverhandlung eines weiteren Nuklearabkommens mit dem Iran.
      
    \section{Punkte zur Diskussion}

In der Vorbereitung und in der Diskussion im Gremium sollte der Sicherheitsrat die folgenden Fragen versuchen zu klären: \\

-Ist es in der gegenwärtigen Situation möglich, den JCPOA noch langfristig einzuhalten? Wenn ja, wie und welche Kompromisse sollte man hierfür eingehen? \\ \\
-War der Austritt der USA aus dem JCPOA und die Wiedereinführung von Sanktionen gerechtfertigt? Wenn nicht, welche Aspekte des Völkerrechts haben die USA mit ihren Handlungen verletzt und wie kann die Situation gelöst werden? \\ \\
-Ist das aktuelle Kontrollregime, das die Einhaltung des JCPOA durch den Iran überprüft, ausreichend oder waren die Kritikpunkte der USA gerechtfertigt? \\ \\
-Ist die Einhaltung des JCPOA unter Berücksichtigung der Resolution 2231 des Sicherheitsrates als völkerrechtlich verbindlich anzusehen? Wenn ja, wie wäre mit dem Vertragsbruch der USA umzugehen? \\ \\
-Sollte in der Zwischenzeit, solange sich der Iran weiterhin an den JCPOA hält, die schrittweise Aufhebung von Sanktionen weiterhin erfolgen oder sollte der JCPOA aufgegeben oder ausgesetzt werden und eine Neuverhandlung zusammen mit der USA angestrebt werden?


\vspace{1cm}

  \section{Weiterführende Links und wichtige Dokumente}
  
\href{http://www.un.org/depts/german/sr/sr_15/sr2231.pdf}{-Resolution 2231 des Sicherheitsrats (Deutsche Fassung)} \\ \\ \href{https://www.securitycouncilreport.org/atf/cf/\%7B65BFCF9B-6D27-4E9C-8CD3-CF6E4FF96FF9\%7D/s_res_2231.pdf}{-Resolution 2231 des Sicherheitsrats (Englische Fassung)} \\ \\ \href{https://fas.org/sgp/crs/nuke/LSB10134.pdf}{-Rechtliche Details zur Resolution (EN)} \\ \\ \href{https://www.state.gov/documents/organization/245317.pdf}{-Text des JPCOA (EN)} \\ \\ \href{https://www.auswaertiges-amt.de/blob/207392/b38bbdba4ef59ede2fec9e91f2a8179b/nvv-data.pdf}{-Atomwaffensperrvertrag (DE)} \\ \\ \href{https://www.theguardian.com/world/live/2018/may/08/iran-nuclear-deal-donald-trump-latest-live-updates}{-''The Guardian'' Artikel zum Austritt der USA (EN)} \\ \\ \href{https://www.zeit.de/politik/ausland/2018-05/iran-donald-trump-kuendigt-ausstieg-aus-atomabkommen-an}{-''Die Zeit'' Artikel zum Austritt der USA (DE)} \\ \\ \href{https://www.nytimes.com/interactive/2018/05/07/world/middleeast/iran-deal-before-after.html}{-''New York Times'' Artikel zur Lage des JCPOAs nach dem Austritt der USA (EN)} \\ \\ \href{http://www.faz.net/aktuell/politik/ausland/einigung-mit-iran-zusammenfassung-des-atomabkommens-13702441.html}{-Zusammenfassung der “Frankfurter Allgemeine” zum JCPOA (DE)} \\ \\ \href{https://www.wko.at/service/aussenwirtschaft/USA-setzt-Iran-Sanktionen-wieder-in-Kraft.html}{-US-Sanktionen gegen Iran (DE)} \\ \\ \href{https://www.un.org/press/en/2018/sc13398.doc.htm}{-Aktuelle Stellung der Mitglieder des Sicherheitsrates (EN)} \\ \\ \href{https://www.armscontrol.org/factsheet/Timeline-of-Nuclear-Diplomacy-With-Iran}{-Zeitlinie bis Oktober 2018 zum Nuklear-Programm Irans (EN)} \\ \\ \href{https://www.mdr.de/nachrichten/politik/ausland/chronologie-atomstreit-mit-iran-100.html}{-Zeitlinie bis Mai 2018 zum Nuklear-Programm Irans (DE)} \\ \\ \href{http://www.bits.de/public/pdf/rr06-1.pdf}{-Artikel zu den nuklearen Programm Irans vom Berliner Informationszentrum für Transatlantische Sicherheit (DE)} \\ \\ \href{https://www.iaea.org/newscenter/focus/iran/iaea-and-iran-iaea-reports}{-Berichte der IAEA zur iranischen Einhaltung ans JCPOA seit 2015 (EN)}
 
\end{document}
