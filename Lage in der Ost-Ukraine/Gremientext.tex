\documentclass[a4paper,11pt]{article}

\usepackage{ngerman}
\usepackage{soul}
\usepackage{mathtools}
\usepackage{amssymb,amsmath,amsfonts}
\usepackage[utf8]{inputenc}
\usepackage{graphicx}
\usepackage{geometry}
\usepackage{float}
\usepackage[german=quotes]{csquotes}
\usepackage{hyperref}
\usepackage{fancyhdr}
\usepackage{gensymb}
\usepackage{units}
\usepackage{hhline}
\usepackage{color}
\usepackage[export]{adjustbox}
\usepackage[nottoc,numbib]{tocbibind}
\usepackage[super,comma]{natbib}
\usepackage{titling}

\geometry{a4paper, left=30mm, right=30mm, top=30mm, bottom=30mm}
\definecolor{pantone294}{cmyk}{1,0.6,0,0.2}

\title{Die Lage in der Ost-Ukraine \\ Der Krieg in Donbass und die Krimkrise}
\author{Vorsitz Sicherheitsrat \\ \\ Santiago Rodriguez}
\date{25.10.2019}
\pagestyle{fancy}
\lfoot{MUN-BW 2020}
\rfoot{Lage in der Ost-Ukraine}

\begin{document}
	\newgeometry{left=14mm, right=13.5mm, top=13.5mm, bottom=30mm}
	\begin{titlepage}
		\thispagestyle{empty}
		\begin{figure}
			\includegraphics[width=31.5mm,right]{./munbwlogo.png}
		\end{figure}
		\vspace*{-43mm}\hspace{-6mm}\textbf{\textcolor{pantone294}{\large{DMUN}}}\\\\\\\\\\
		\textcolor{pantone294}{MUN-BW 2020}\\
		\vspace{30mm}
		\begin{center}
			\textcolor{pantone294}{\huge{MUN-BW 2020}}\\\vspace*{7mm}
			\textcolor{pantone294}{\huge{\textbf{\thetitle}}}\\\vspace*{10mm}
			\textcolor{pantone294}{\theauthor}\\\vspace*{10mm}
			\textcolor{pantone294}{\thedate}\\\vspace*{20mm}
		\end{center}
	\end{titlepage}
	\makeatother
	\restoregeometry
	\newpage
	
	\tableofcontents
\vspace{2cm}
	
	
	\section{Einleitung}
Die Situation in der Ostukraine ist seit Februar 2014 eine der kriegerischen Auseinandersetzung zwischen der ukrainischen Regierung, Russland und den selbsternannten Volksrepubliken Lugansk und Donezk über die völkerrechtliche Zugehörigkeit der östlichen und südöstlichen Gebiete der Ukraine; Krim, Luhansk und Donezk. Ausgelöst durch die Euromaidane Bewegung von 2013 und vertieft durch die Annexion der Halbinsel Krim durch Russland in März 2014, die Krise um die Ostgebiete (bzw. auch als Oblasten bezeichnet) der Ukraine, Luhansk und Donezk, läuft bis zum heutigen Tage schon fünf Jahre lang in der Form eines bewaffnetem Konfliktes zwischen der ukrainischen Regierung in Kiew und den Separatisten auf den jeweiligen Oblasten fort. Inoffiziell unterstützt von der russischen Regierung wollen die Separatisten aus den beiden Gebieten die komplette politische und wirtschaftliche Unabhängigkeit von der Ukraine erzwingen sowie anstelle der ukrainischen Verwaltung die selbstständigen Volksrepubliken Lugansk und Donezk ausrufen, die zudem in einer engeren Beziehung zu Russland stehen sollen. \\ \\ Seit dem Kriegsanfang in 2014 sind mehrere Waffenstillstände zwischen den beiden Konfliktparteien verhandelt aber auch wiederholt verletzt worden, was zum Einsatz von immer schwereren Waffen und zur Entstehung von weiteren tödlichen Auseinandersetzungen auf den Gebieten der Ostukraine geführt hat. Aktuell sind mehr als 13.000 Menschen während den Kämpfen in der Ostukraine ums Leben gekommen und bis zu 30.000 anderweitig verletzt worden. Einen langfristigen Waffenstillstand zu verhandeln, der Verhandlungen zwischen den Konfliktparteien ermöglicht und weitere Todesopfer vermeidet, ist somit ein wichtiger Meilenstein für die Arbeit der Vereinten Nationen in den beiden Oblasten von Luhansk und Donezk. In der Halbinsel Krim ist die Situation wiederum aber eine deutlich andere. Der politische Status der Halbinsel Krim, die in März 2014 völkerrechtswidrig von Russland als eins seiner Föderationssubjekte annektiert wurde, ist bis heute noch auf internationaler Ebene umstritten und wird von der Gesamtheit der internationalen Staatengemeinschaft bis auf Russland als Teil des ukrainischen Verwaltungsgebiet anerkannt. Trotz dieser Anerkennung unterliegt das Gebiet der Halbinsel Krim aktuell russischer Verwaltung, nachdem seit dem 27. Februar 2014 das dort stationierte, russische Militär anfing, die politische Kontrolle des Gebiets durch den Einzug in strategische Verwaltungsgebäude sowie andere, verdeckte Interventionen an prorussische Stimmen zu reißen. Insbesondere durch die militärische Anwesenheit Russlands im Schwarzen Meer und in der Großstadt der Halbinsel, Sewastopol, sowie durch die Abwesenheit bewaffneter Kämpfe auf dem Gebiet aufgrund der kompletten russischen Kontrolle über das Territorium, wird die aktuelle russische Besetzung unvermeidbar deutlich gemacht. Diese Besetzung der Halbinsel Krim stellt insbesondere eine Gefahr für die internationale Sicherheit dar, da das akzeptieren selbiger einen präzedenzlosen, erfolgreichen Verstoß gegen den Artikel 2, Absatz 4 der Charta der Vereinten Nationen bedeuten würde, die bisher die territoriale Integrität jeglicher Mitglieder der internationalen Staatengemeinschaft völkerrechtlich garantiert hatte. \\ \\ Ein solches Scheitern würde insofern nicht nur ein Scheitern des Völkerrechts bedeuten, sondern auch ein Scheitern der Charta der Vereinten Nation sowie der Vereinten Nationen selbst, da diese auf den Grundkonzepten des Völkerrechts aufgebaut sind. Aus diesem Grunde ist eine Lösung des bewaffneten Konfliktes in Luhansk und Donezk ohne den Status von Krim, Luhansk oder Donezk als Teil der Ukraine aufzugeben von höchster Priorität für den Sicherheitsrat und Ziel der Verhandlungen zu diesem Tagesordnungspunkt.  

	\section{Hintergrund und Grundsätzliches}
Ihren Anfang nahm der Konflikt mit der sogenannten Euromaidanen Bewegung von Ende 2013, die eine nähere Beziehung der Ukraine mit Europa forderte. Ursache dafür war, dass sich die damalige ukrainische Regierung unter den Präsidenten Wiktor Janukowytsch in November desselben Jahres dazu geweigert hatte, einen bisher von der pro-europäischen Wählerschaft vorgesehenen Assoziierungsabkommen mit der europäischen Union zu unterzeichnen. Das Janukowytsch Kabinett hatte bereits zuvor wiederholt die Unterzeichnung verschoben und lag ebenfalls unter Verdacht für verbreitete und systematische Korruption in ihren Kreisen. Dennoch waren die langfristigen Bedenken der Regierung gegenüber dem Assoziierungsabkommens auch nicht unbegründet; eine Unterzeichnung des Vetrags hätte nämlich eine Distanzierung von Russland und der Eurasischen Wirtschaftsunion bedeutet und riskierte aufgrund des großen ethnischen Anteil an Russen- insbesondere in den Oblasten von Krim, Luhansk und Donazk- und der langen Geschichte der Ukraine mit Russland, sowohl während der Zeit der Sowjetunion als auch früher, die mehr als 8 Millionen Russen innerhalb der Ukraine von dem Rest der nicht russischen Bevölkerung zu entfremden. Die pro-europäische Euromaidane Bewegung forderte jedoch noch bis Februar 2014 die Absetzung des Präsidenten Janukowytsch und die Unterzeichnung des Assoziierungsabkommen mit der Europäischen Union. Nach langen, unter anderem auch gewalttätigen Protesten gelang es der Bewegung am 22. Februar den umstrittenen Präsidenten zur Flucht aus dem Lande zu zwingen und eine Übergangsregierung an seiner Stelle an die Macht zu bringen, die einerseits das politische Gerüst des Assoziierungsabkommens mit der Europäischen Union am 21. März unterzeichnete sowie neue Wahlen für dasselbe Jahr ansetzte. \\ \\ Die Übergangsregierung wurde aber von der russischen Föderation, die kurz zuvor auch den geflüchteten Präsidenten der Ukrainee Asyl erteilt hatte, als illegitim bezeichnet und nicht anerkannt. Der Kremlin warf der Euromaidanen Bewegung und der vom ukrainischen Parlament aufgestellten Übergangsregierung vor, die politische Macht über das Land in einem illegalen Putsch an sich gerissen zu haben um die Stimme der russischen Minderheiten, die für Janukowytsch während den Wahlkampf von 2010 gestimmt hatten, zu unterdrücken. Diese Ansicht wurde aber von dem Rest der internationalen Staatengemeinschaft bestritten, die sowohl die Entmachtung des Präsidenten Janukowytsch als auch die Machtübernahme der Übergangsregierung offiziell für verfassungsgemäß und legitim erklärten. Die Spannung zwischen den beiden Parteien stieg dann anschließend am 27. Februar weiter als russische Truppen, die zu dem Zeitpunkt auf der Halbinsel von Krim in russischen Militärbasen stationiert gewesen waren, anfingen, mithilfe verdeckter Interventionen und Eingriffe in strategische Standorte die Verwaltungskontrolle über die Halbinsel an pro-russische Politiker zu reißen. Diese riefen dann schließlich am 16. März 2014 einen verfassungswidrigen Referendum über den politischen Status der Krim aus, bei dem nach offiziellen Quellen der russischen Verwaltung eine Mehrheit von 96,77\% der Bevölkerung in Krim bei 83,1\% Wahlbeteiligung für die Unabhängigkeit von der Ukraine und die Eingliederung in die russische Föderation stimmte. Inwiefern diese Stimmverhältnisse der Realität entsprachen wurde zwar von zahlreichen Außenseitern in Frage gestellt und die Verfassungswidrigkeit des Referendums ebenfalls entschieden verurteilt, doch die russische Föderation fuhr trotz den fast einstimmigen Aufschrei der internationalen Staatengemeinschaft mit der offiziellen Annektierung der Krim als eins seiner Föderationssubjekte am 21. März fort. \\ \\ Parallel zur Aufruhr an der Halbinsel Krim wurde auch in den beiden Oblasten von Donazk und Luhansk von lokalen Separatisten versucht, mithilfe russischer Unterstützung zwei selbstständige Volksrepubliken und die Unabhängigkeit von der Ukraine auszurufen. Der Konflikt eskalierte seit dessen Anfang in März zu einem uneingeschränkten Krieg zwischen den Separatisten und der ukrainischen Regierung in Juli. Im selben Monat wurde sogar im Kreuzfeuer zwischen beiden Kampfparteien der Flugzeug von Malaysia Airlines MH17 von russischen Streitkräften mithilfe einer Flugabwehrrakete gestürzt, nachdem dieser ein Kampfgebiet im Südosten der Ukraine überflogen hatte. Bei diesem Unfall starben fast 300 Zivilisten und es waren geschätzt bis zu 100 russische Soldaten beim Transport und der Abfeuerung der Rakete beteiligt gewesen. Der Sicherheitsrat versuchte, einen Sondertribunal zur Untersuchung des Kriegsverbrechens gegen zivile Opfer einzurichten, doch die Resolution wurde durch ein russisches Veto blockiert. Es wurden ebenfalls weitere Resolutionen des Sicherheitsrates gegen die Gültigkeit des Krim Referendums, die Annexion durch Russland und die Verurteilung der kriegführenden Separatisten von den russischen Entsandten blockiert. Seit Anfang des Konfliktes hat sich die Arbeit des Sicherheitsrates somit mehr darauf gerichtet, einen Waffenstillstand in der Region in die Wege zu leiten. Da die russische Föderation mit einem solchen Verfahren einverstanden ist, konnten auch mehrere Resolutionen wie Res. 2202 zur Annahme der Minsker Abkommen oder Res. 2166 zum Bedauern des Absturzes von Flug MH17 ohne Deutung von Schuldigen verabschiedet werden. Eine Resolution, die aber zumindest annähernd auf eine mögliche Lösung des Konfliktes hindeutet, konnte bisher nicht verhandelt werden und selbst die abgestimmten Waffenstillstände zur Implementierung der Minsker Abkommen wurden bis zum heutigen Datum wiederholt von beiden Konfliktparteien verletzt.
  \newpage		

    \section{Aktuelles}
   Aktuell hat sich die Situation in den Oblasten von Donezk und Luhansk zu eine der Abwechslung zwischen Waffenstillständen und kriegerischen Auseinandersetzungen entwickelt. Das Ziel der meisten diplomatischen Akteure ist es, die Minsker Abkommen -im allgemeinen Sprachgebrauch auch abwechselnd als Minsk I und  Minsk II bezeichnet- zu implementieren und das bewaffnete Konflikt endgültig zu beenden. Die Minsker Abkommen wurden jeweils in 2014 und 2015 in der Hauptstadt von Weißrussland, Minsk, vereinbart, wobei aber Minsk I vielmehr eine schriftliche Zusammenfassung von Punkten ist, die in Beratungen zur Lahmlegung des Konfliktes vorgeschlagen wurden, und Minsk II das tatsächliche Vertrag zwischen der Ukraine, Russland, Deutschland, Frankreich und der Vereinigten Staaten ist, in denen sie sich zur Einhaltung von eben diesen Punkten einigten. Hierbei sahen sowohl Minsk I als auch Minsk II die Einstellung eines von der OSZE (Organisation für Sicherheit und Zusammenarbeit in Europa) übersehenen Waffenstillstandes und den Rückzug ukrainischer Truppen und russischen Schwerwaffen aus der Front zwischen den Gebieten vor. Außerdem sollten die beiden Oblaste von Luhansk und Donezk erneut in die Ukraine eingegliedert und durch ein sonderlich eingeführtes Selbstverwaltungssystem mitsamt neuer, lokaler Wahlen innerhalb der territorialen Integrität der Ukraine sein Selbstbestimmungsrecht in der Form einer von Kiew unabhängigeren Selbstverwaltung ausüben. \\ \\ Die Umsetzung dieser Punkte scheiterte aber wenige Tage nach der Unterzeichnung von Minsk II am 5.9.2015 als der Waffenstillstand gebrochen und das blutige Konflikt erneut zum leben erwacht wurde. Auch die Situation auf der Halbinsel Krim ist seit dessen Annektierung durch Russland dieselbe geblieben und trotz wirtschaftlicher Sanktionen der Europäischen Union unterliegt die Halbinsel heutzutage immer noch russischer Verwaltung. Anfang November 2019 fingen jedoch nach erneuten Gesprächen der ukrainischen Regierung mit Russland sowohl ukrainische Truppen als auch separatistische Streitkräfte aus der Front der Gebiete von Zolote und Petrivske in Luhansk wegzuziehen. Weitere Gespräche um den Konflikt endgültig zu beenden befinden sich zurzeit im Gange, doch ob dieser Waffenstillstand endgültig bestehen bleiben wird ist unklar. \\ \\ Zurzeit hängen die Verhandlungsgespräche zwischen der Ukraine und Russland im wesentlichen davon ab, ob sie die unter den Minsker Abkommen vereinbarten Maßnahmen noch implementieren wollen oder sich nun auf die Erstellung eines neuen Abkommens einlassen möchten. Eines der mehreren Kritikpunkte von diplomatischen Versandten gegen die Minsker Abkommen waren die Verpflichtungen, an denen sich Russland, der vermeintliche Hauptunterstützer der Rebellen in Luhansk und Donezk, einzuhalten habe; nämlich keine. Denn obwohl das Abkommen den Unterzeichnerstaaten mehrere Richtlinien vorschreibt um den Konflikt in der Ostukraine zu entschärfen, so sind all diese Vorschriften auf keinster Weise verbindlich für die russische Föderation oder andere beteiligte Konfliktparteien formuliert und bis zu einem hohen Grad frei interpretierbar von seitens der Unterzeichner. Auch hat der russische Kreml öffentlich wiederholt verleugnet, dass Russland die separatistischen Streitkräfte und Milizen mit Waffen, Geldern und sogar Soldaten während den Konflikt unterstützt hat, und bei Sitzungen des Sicherheitsrates lediglich das Recht zur Selbstbestimmung der Völker wiederholt betont sowie sämtliche, kritisch betonte Resolutionen blockiert. 

  \newpage
    
    \section{Probleme und Lösungsansätze}
Für die internationale Staatengemeinschaft ist die Frage rund um die Oblasten Luhansk, Donezk und Krim eine der territorialen Integrität der Ukraine, die nach dem Völkerrecht und der Charta der Vereinten Nationen gewährleistet werden muss. Eine Lösung beinhaltet somit für den Sicherheitsrat in dessen Rolle als Organ der Vereinten Nationen die Rückgabe der Souveränität über die umstrittenen Territorien an die Ukraine, sowie die endgültige Beendigung des Konfliktes. Die entgegengesetzte Stellung der russischen Föderation erschwert die Verhandlung und Umsetzung einer solchen Lösung aber aufgrund von dessen notwendiger Zustimmung bei jeglichen Beschlüssen des Sicherheitsrates. Die Regierung in Moskau erwidert das Argument der territorialen Integrität mit dem des Selbstbestimmungsrechts der Völker, die ihrer Meinung nach in einem Referendum zum Status der Gebiete zu Wort gelassen werden sollte, so wie es beim Referendum in Krim anscheinend der Fall gewesen sein soll. Hinter diesen anscheinend völkerrechtlichen Argument liegen aber nach Meinung amerikanischer Botschafter territoriale und militärisch-strategische Gründe Russlands, die sie unter allen Umständen und selbst unter einer Verletzung des Völkerrechts sichern wollen. \\ \\ Aufgrund der entgegengesetzten Einstellungen der Konfliktparteien sind Kompromisse wie die Minsker Abkommen somit aktuell nur schwer erreichbar und selbst dann ist dessen Einhaltung noch komplizierter aufgrund der nicht verbindlichen Natur, die ein solches Abkommen benötigt um von allen Konfliktparteien unterzeichnet werden zu können. Die russische Föderation leugnet außerdem dessen Beteiligung am Konflikt in Luhansk oder Donezk und lehnt in der Öffentlichkeit fest ab, die separatistische Streitkräfte dort mit Waffen und anderen Mitteln versorgt zu haben, obwohl schon mehrmals Waffensysteme russischer Herkunft so wie das Raketensystem der den Flug MH17 abschoss nachgewiesen worden waren. Selbst in den Minsker Abkommen findet man keinen Hinweis auf die russische Anwesenheit in ukrainisches Territorium, was dessen fehlende Verbindlichkeit gegenüber Russland betont. Dennoch waren die Minsker Abkommen insofern ein bedingter Erfolg dadurch, dass durch die Verhandlung dieser gemeinsamen Richtlinien eine Eskalation des Konfliktes wie in Syrien vermieden werden konnte. Einen Waffenstillstand zu verhandeln, der den Konflikt aufs erste entschärft bis eine endgültige Lösung bezüglich zum Status der Krim, sowie Luhansk und Donezk verhandelt werden kann, wäre somit auch ein völlig befriedigendes Ansatz für den zukünftigen Umgang mit dem Konflikt und könnte entweder im Rahmen der Implementierung der Minsker Abkommen oder von Neuverhandlungen vereinbart werden. \\ \\
      
    \section{Punkte zur Diskussion}
-Inwiefern sind die Handlungen der russischen Föderation als völkerrechtswidrig einklagbar und wie soll der Sicherheitsrat damit umgehen? \\ \\ -Sollten die Minsker Abkommen erneut implementiert werden oder stattdessen Neuverhandlungen erzielt werden? Falls letzteres der Fall ist, welche Punkte müssen noch angesprochen werden und welche können bereits aus der Vorlage der Minsker Abkommen entnommen werden? \\ \\ -Soll die OSZE weiterhin für die Überwachung der Konfliktregionen verwendet oder u.U. andere Alternativen wie der Versand von UN-Blauhelmtruppen zur Überwachung erkundet werden? \\ \\ -Welcher politische Status soll den Gebieten von Luhansk, Donezk und Krim vorübergehend bis zur Lösung des Konfliktes erteilt und von der internationalen Staatengemeinschaft anerkannt werden? Sollen diese ukrainischer Verwaltung unterliegen oder der Verwaltung der selbsternannten Republiken? \\ \\ -Müsste eine unabhängige Untersuchungskommission zur Untersuchung der russischen Einflussnahme an dem Konflikt um Donezk und Luhansk beauftragt werden? Sollte evtl. auch zur Klärung der völkerrechtlichen Frage um Luhansk und Donezk der internationale Gerichtshof mit einberufen werden? 



  \section{Weiterführende Links und wichtige Dokumente}
  
\href{https://www.welt.de/newsticker/dpa_nt/infoline_nt/thema_nt/article131986171/Das-Minsker-OSZE-Protokoll-fuer-eine-Feuerpause.html}{Richtlinien des Minsker Abkommens, übersetzt aus dem Russischen [Deutsch]} \\ \href{http://www.securitycouncilreport.org/atf/cf/\%7B65BFCF9B-6D27-4E9C-8CD3-CF6E4FF96FF9\%7D/s_res_2202.pdf}{Resolution 2202 des Sicherheitsrates [Englisch]} \\ \href{http://www.securitycouncilreport.org/atf/cf/\%7B65BFCF9B-6D27-4E9C-8CD3-CF6E4FF96FF9\%7D/s_res_2166.pdf}{Resolution 2166 des Sicherheitsrates [Englisch]} \\ \href{https://www.bbc.com/news/world-europe-18010123}{Zeitlinie des bewaffneten Konfliktes [Englisch]} \\ \href{https://www.amnesty.de/jahresbericht/2018/ukraine}{Annual Report von Amnesty International zur Lage in der Ukraine 2017/2018 [Englisch/Deutsch]} \\ \href{http://www.securitycouncilreport.org/atf/cf/\%7B65BFCF9B-6D27-4E9C-8CD3-CF6E4FF96FF9\%7D/s_res_2202.pdf}{Resolution 2202 des Sicherheitsrates [Englisch]} \\ \href{https://www.zeit.de/politik/ausland/2016-02/minsk-2-ukraine-putin-russland-grenze-abkommen-krieg-frieden}{Artikel zu dem Minsker Abkommen nach einem Jahr [Deutsch]} \\
 
\end{document}
