\documentclass[a4paper,11pt]{article}

\usepackage{ngerman}
\usepackage{soul}
\usepackage{mathtools}
\usepackage{amssymb,amsmath,amsfonts}
\usepackage[utf8]{inputenc}
\usepackage{graphicx}
\usepackage{geometry}
\usepackage{float}
\usepackage[german=quotes]{csquotes}
\usepackage{hyperref}
\usepackage{fancyhdr}
\usepackage{gensymb}
\usepackage{units}
\usepackage{hhline}
\usepackage{color}
\usepackage[export]{adjustbox}
\usepackage[nottoc,numbib]{tocbibind}
\usepackage[super,comma]{natbib}
\usepackage{titling}

\geometry{a4paper, left=30mm, right=30mm, top=30mm, bottom=30mm}
\definecolor{pantone294}{cmyk}{1,0.6,0,0.2}

\title{Maßnahmen gegen die globale Terrorismus-Finanzierung}
\author{Vorsitz Ausschuss zur Bekämpfung des Terrorismus \\ \\ Santiago Rodriguez}
\date{25.10.2019}
\pagestyle{fancy}
\lfoot{MUN-BB 2019}
\rfoot{Maßnahmen gegen Terror-Finanzierung}

\begin{document}
	\newgeometry{left=14mm, right=13.5mm, top=13.5mm, bottom=30mm}
	\begin{titlepage}
		\thispagestyle{empty}
		\begin{figure}
			\includegraphics[width=31.5mm,right]{./MUNBB_Logo.png}
		\end{figure}
		\vspace*{-43mm}\hspace{-6mm}\textbf{\textcolor{pantone294}{\large{DMUN}}}\\\\\\\\\\
		\textcolor{pantone294}{MUN-BB 2019}\\
		\vspace{30mm}
		\begin{center}
			\textcolor{pantone294}{\huge{MUN-BB 2019}}\\\vspace*{7mm}
			\textcolor{pantone294}{\huge{\textbf{\thetitle}}}\\\vspace*{10mm}
			\textcolor{pantone294}{\theauthor}\\\vspace*{10mm}
			\textcolor{pantone294}{\thedate}\\\vspace*{20mm}
		\end{center}
	\end{titlepage}
	\makeatother
	\restoregeometry
	\newpage
	
	\tableofcontents
\vspace{2cm}
	
	
	\section{Einleitung}
Für den Ausbau ihrer Netzwerke sowie die Logistik und Ausführung von Anschlägen oder sonstigen Aktivitäten müssen Terrororganisationen über Finanzierungsmöglichkeiten verfügen. Diese ermöglichen es ihnen, illegale Aktivitäten durchzuführen, ohne dabei die Aufmerksamkeit der zuständigen Behörden zu erregen. Die Schlagkraft terroristischer Organisationen hängt somit direkt von ihrer Fähigkeit ab, Geld aus unterschiedlichen Quellen zu erlangen und möglichst effektiv entlang ihrer Finanzstruktur zu verwalten, sodass dieses für mögliche Einsätzen und Terroranschläge zur Verfügung steht. Finanzen können sowohl durch rechtlich legale Mittel als auch mithilfe von illegalen Prozeduren wie z.B. Geldwäsche verwaltet werden. Somit ist das offensichtliche Ziel der Bekämpfung von Finanzierungsmethoden auch die Identifizierung, Lahmlegung und endgültige Unterbindung solcher Mittel. So wurden weltweit bisher Vermögenswerte von mehr als 140 Millionen US-Dollar eingefroren, die mit organisiertem Terrorismus in Verbindung gebracht werden. 

	\section{Hintergrund und Grundsätzliches}
Grundlage zur Bekämpfung der Terrorismusfinanzierung ist eine tiefgründige Auseinandersetzung mit den unterschiedlichen Finanzstrukturen, Geldquellen und Überweisungsmöglichkeiten von terroristischen Gruppierungen, die sowohl auf legalen als auch auf illegale Methoden beruhen. Relevant für die Finanzierung von Terrororganisationen sind vor allem die Quellen flüssigen Kapitals; gelänge es, diesem ständigen Geldstrom Einhalt zu gebieten, bevor er in den weltweiten Finanzmarkt anonym eingeführt und weitergeschickt werden kann, wäre dem internationalen Terrorismus einer seiner wichtigsten Grundpfeiler entzogen. Die Komplexität dieses Vorhabens ist aber spätestens seit den terroristischen Anschlägen auf das World Trade Center am 11. September 2001 bekannt, denn ein Großteil der Terrorismusfinanzierung beruht auf nicht dokumentierten, vollständig illegalen Aktivitäten, wie etwa dem Drogenhandel oder der organisierten Kriminalität (z.B. Anschläge, Überfälle, Entführungen gegen Lösegelder). Die wenigen legalen Geldquellen des organisierten Terrorismus ergeben sich im Wesentlichen aus angeblichen Spenden für das Gemeinschaftswohl, politisch motivierten Zuwendungen aus Drittländern und unrechtmäßigen Einzahlungen von Sozial- und Versicherungsgeldern. So wurde zum Beispiel Nichtregierungsorganisationen wie der International Islamic Relief Organisation (IIRO) oder öffentlichen Moscheen mehrmals vorgeworfen, an wohltätige Zwecke gerichtete Spenden in Wirklichkeit zur Finanzierung des Islamischen Terrorismus verwendet zu haben. Teilweise sind auch ganze Staaten wie die islamische Republik Iran oder die Demokratische Volksrepublik Korea von der internationalen Staatengemeinschaft dafür kritisiert worden, Terrororganisationen für außenpolitische Zwecke finanziell unterstützt zu haben.
\\ \\
Um ihr Kapital sicher und im Geheimen bewegen zu können, wird eine Vielzahl an legalen und illegalen Methoden genutzt. Unter den Illegalen überwiegt das Prinzip der Geldwäsche, die zum Einschleusen von illegal erwirtschafteten Geldern aus den erwähnten Geldquellen in das legale Finanzsystem dient. Dabei besteht das übliche Prozedere immer aus denselben drei Schritten: Einspeisen, Verschleiern und Integrieren der illegalen Vermögenswerte. Beim Einspeisen wird das durch Straftaten erlangte Kapital in das Finanzsystem mithilfe von Zahlungen, Erwerbungen und Rechnungen eingeführt. Ein Beispiel hierfür ist Geldwäsche durch flexibel weiter verkaufbare Immobilien, die offiziell für einen Betrag von bspw. 200.000€ erworben und dann für 250.000€ weiterverkauft werden. Hierbei wird in Wirklichkeit eine weit höhere Summe (z.B. 300.000€) in der Form von Bargeldern an den ursprünglichen Eigentümer gezahlt; durch den erneuten Verkauf für 250.000€ wird die saubere Herkunft der 50.000€ vorgespielt. Mithilfe mehrerer, verhältnismäßig kleiner Anwendungen dieses Prinzips wird Schwarzgeld langsam in das legale Finanzsystem eingespeist und somit für den nächsten Schritt, der Verschleierung, bereitgestellt. An dieser Stelle wird die Herkunft des Geldes mithilfe einer Vielzahl von Transaktionen verschleiert, um somit sicherzustellen, dass Finanzbehörden die ursprüngliche Einspeisungstransaktion nicht ohne viel Aufwand nachvollziehen können. Sobald dieser zweite Schritt erfolgt ist, liegt also der Integrierung der nun legal scheinenden Schwarzgelder in den Wirtschaftskreislauf nichts mehr im Wege. Diese können wie andere Geldmengen investiert, angelegt und für den legalen Erwerb von Gütern verwendet werden.
\\ \\
Eine weitere Methode des Geldtransfers, der in der arabischen Welt weit verbreitet und aufgrund mangelnder Kontrollmöglichkeit in mehreren Staaten bereits verboten ist, ist das sogenannte Hawala-System. Hierunter versteht man ein anonymes Finanztransfersystem, bei dem zwei Transferhändler, die Hawaladars, an unterschiedlichen Orten für zwei verschiedene Personen arbeiten. Das System beruht vor allem auf dem Vertrauen aller Beteiligten, denn das an sich überwiesene Geld bewegt sich im Grunde nicht. Die Person, die das Geld überweisen will, gibt lediglich den zu überweisenden Betrag samt eines Codes an seinen Hawaladar, der dann wiederum den anderen Hawaladar beauftragt, dem Empfänger mit dem gleichen Code genau dieselbe Geldmenge auszuzahlen, ohne dass sich das Geld aber wirklich zwischen den Hawaladars bewegt. Dies unterbindet jegliche Kontrollmöglichkeiten, da nur die Hawaladars selbst über die eigentliche Herkunft und den Betrag der überwiesenen Vermögenswerte wissen und diese Information nur auf freiwilliger Basis registrieren und bereitstellen müssen. Da dies oft nicht geschieht, ist das Hawala-System sehr beliebt im organisierten Terrorismus und ein besonders effektives Mittel der Finanzierung. Die Identifizierung und Kontrolle solcher Methoden ist somit unumgänglich für die Bekämpfung terroristischer Aktivitäten und die endgültige Unterbindung weiterer illegaler Finanzierungsmöglichkeiten. 


    \section{Aktuelles}
   Gegenwärtig ist das Spektrum unterschiedlicher Geldquellen bei der Finanzierung von Terrororganisationen außerordentlich weit. Es reicht vom illegalen Handel mit gewöhnlichen Rohstoffen wie Erdöl bis hin zum exotischen Schwarzmarkt mit Elfenbein. Auch durch andere illegale Aktivitäten wie Drogen- und Menschenhandel werden große Geldsummen des terroristischen Budgets generiert. Die Maßnahmen zur Bekämpfung solcher Geldquellen muss daher angepasst werden. Dies wird weiterhin erschwert, da jede einzelne Terrororganisation unterschiedlich stark von den jeweiligen Geld- und Personalquellen abhängig ist. So hängen muslimische Terrororganisationen wie der IS oder die Hisbollah im Nahen Osten wesentlich stärker vom Handel mit Erdöl ab als beispielsweise lateinamerikanische Guerillaorganisationen, die deutlich mehr auf illegale Unternehmen wie dem Drogenhandel oder dem unerlaubten Abholzen von Wäldern angewiesen sind, um ihre finanzielle Struktur aufrecht zu erhalten. Schließlich unterstützen und finanzieren auch von der Weltgemeinschaft anerkannte Staaten terroristische Gruppierungen für innen- oder außenpolitische Zwecke, ein Beispiel hierfür ist die Unterstützung der Islamischen Republik Irans von Terrorgruppierungen wie Hisbollah oder Hamas. Aufgrund dieser großen Vielfalt an unterschiedlichen Geldquellen haben sich die Maßnahmen gegen die globale Terrorismusfinanzierung bisher vor allem darum bemüht, die von den Terrororganisationen angewandten Methoden der Geldwäsche zu identifizieren und lahmzulegen. Im Zuge dessen ist das Hawala-System bereits in mehreren Ländern, beispielsweise Indien und Pakistan, für illegal erklärt worden. Auch werden weiterhin mehrmals im Jahr Bankkonten und Geldanlagen, deren terroristische Verwendung identifiziert wurde, eingefroren.
\\ \\
International gibt es bereits mehrere Beschlüsse, die der Koordination solcher Maßnahmen dienen: Die Resolution A/RES/54/109 der Generalversammlung der Vereinten Nationen vom 25. Februar 2000 zur Bekämpfung der Terrorismusfinanzierung wurde mit dem Ziel verabschiedet, den illegalen Status der Finanzierung von Terrorismus auf internationaler Ebene allgemein anzuerkennen. Die Resolution bezieht sich aber konkret auf die Bekämpfung von illegalen Geldern bei grenzüberschreitenden Überweisungen, wodurch die strafrechtliche Verfolgung jeweils auf das Land entfällt, an dessen Grenze das Geld ursprünglich aufgehalten wird. Die Resolutionen des Sicherheitsrates S/RES/1267/1999 und S/RES/1373/2001 forderten die Einfrierung von Geldern, die als Teil des Al-Qaida- und Taliban-Netzwerkes identifiziert worden waren und bestellten zudem den 1267-Ausschuss sowie den Ausschuss zur Bekämpfung des Terrorismus (Counter-Terrorism Committee – CTC) ein. Nach welchen Regeln und Maßstäben die Verfolgung solcher Gelder schließlich geregelt werden soll, wird jedoch ebenso wenig geklärt wie die Koordination der einzelnen, für den Kampf gegen globale Terrorismusfinanzierung zuständigen Organisationen. Eine deutlich breitere Struktur zur Bekämpfung der Terrorismusfinanzierung bilden hingegen Initiativen wie die Financial Action Taskforce (FATF) und das International Money Laundering Information Network (IMOLIN), die sowohl gemeinsame als auch international anerkannte Standards zur Bekämpfung globaler Finanzstrukturen von Terrororganisationen zur Verfügung stellen (so z.B. die 40+9 Empfehlungen der FATF, ein grundlegender Rahmen zur Aufspürung, Prävention und Bekämpfung von Geldwäsche und Terrorismusfinanzierung). Diese Standards sind jedoch nicht verpflichtend in die Regelungen der in den einzelnen Staaten zuständigen Behörden eingepflegt worden. Es besteht somit auch keine Garantie der globalen Einhaltung. Auch ist bisher keine offizielle Aufforderung dazu ergangen, weshalb sich hier ein weiterer Punkt in der Bekämpfung terroristischer Finanzmittel ergibt, bei dem noch wesentlicher Verbesserungsbedarf besteht. 


    \section{Probleme und Lösungsansätze}
Die Etablierung einer globalen Zusammenarbeit zwischen den für die Bekämpfung der Terrorismusfinanzierung zuständigen Behörden muss aktiv eingefordert werden. Aufgrund der Unterschiede bei der Verbreitung von Informationen über terroristische Gruppierungen liegt zurzeit das Problem vor, dass es keine bzw. ungenügende Kooperation zwischen den zuständigen Behörden gibt. Gegenseitiger Austausch von Informationen ist daher notwendig; hierbei müssen alle Ebenen (Behörden, Bankenaufsichten und Regierungen) auf ein gemeinsames Niveau gebracht werden, auf dem der Austausch von Informationen bezüglich eventueller Maßnahmen so frei wie möglich stattfinden kann. Hilfreich wäre eine international anerkannte Regelung zum Austausch von Informationen. Diese wäre gegebenenfalls mithilfe einer Sonderorganisation der Vereinten Nationen im Vorbild der FATF realisierbar. Auch die international verpflichtende Anerkennung und Implementierung der 40+9 Empfehlungen könnte diskutiert und auf Verbindlichkeit besprochen werden.
\\ \\
Ein weiterer Diskussionspunkt ist zudem, wie mit Staaten und Nichtregierungsorganisationen, die aktiv Terrorismus finanzieren, umzugehen sei. Im Rahmen dessen könnte eine Kontrolle der wirtschaftlichen Aktivitäten von verdächtigen Organisationen in Frage kommen, wobei aber beachtet werden muss, dass dadurch die Arbeit tatsächlich gemeinnütziger Organisationen nicht erschwert und die Sicherheit ihrer Spender weiterhin gewährleistet wird.
\\ \\
Es ist nicht die Aufgabe des Ausschusses, konkrete Verstöße zu identifizieren und Anschuldigungen zu äußern, sondern globale Strukturen zur Bekämpfung von Terrorismusfinanzierung aufzubauen. Die sofortige Einstellung direkter Finanzierung von terroristischen Gruppierungen seitens staatlicher Akteure kann nicht direkt gefordert werden. Auch können bei fehlender Kooperation keine Sanktionen als Gegenmaßnahmen angedroht werden. Als ein dem Sicherheitsrat untergeordnetes Gremium ist der Ausschuss zur Bekämpfung von Terrorismus unter Kapitel VII der Charta der Vereinten Nationen nicht dazu gedacht, als sanktionierender Körper in die völkerrechtliche Souveränität sicherheitsgefährdender Staaten einzugreifen. Bei den Resolutionen des Ausschusses ist es notwendig, dass für die spätere Abstimmung über die Resolution im übergeordneten Gremium, dem Sicherheitsrat, die Zustimmung aller ständigen Mitglieder vorhanden ist. Aber selbst ohne diese weitreichenden Befugnisse ist es notwendig, mit äußerstem diplomatischem Geschick zu debattieren. Da es keine international anerkannte Definition für Terrorismus gibt, kann nicht eindeutig einem Staat oder einer Nichtregierungsorganisation vorgeworfen werden, terroristische Gruppierungen zu unterstützen. Es wäre lediglich möglich, den Organisationen, die unterstützt werden, gut begründet Angriffe auf das Leben anderer Menschen oder die Verletzung körperlicher Unversehrtheit nachzuweisen. Im Mittelpunkt der Diskussion sollte aber immer die Identifizierung der Vermögenswerte von Terrororganisationen sowie die Kontrolle internationaler Finanztransaktionen stehen. Die größte Herausforderung wird in der Debatte die Vermittlung zwischen verschiedensten nationalen Systemen und Interessen sein. Damit die vereinbarten Maßnahmen Wirkung entfalten können, müssen sie zunächst international akzeptiert werden und in nationales Recht umgewandelt werden, denn sowohl die Strafverfolgung als auch die Finanzmarktkontrolle obliegen nationalen Autoritäten. 
    \section{Punkte zur Diskussion}
- Was wären Vor- und Nachteile einer globalen Zusammenarbeit bei der Bekämpfung von Terrorismusfinanzierung? \\ \\ - Wie kann die Kooperation der zuständigen Behörden verbessert werden? \\ \\ - Kann eine UN Behörde hier Aushilfe schaffen? Mit welchen Befugnissen sollte diese ausgestattet werden? \\ \\ - Welche internationalen Maßstäbe müssen etabliert werden, um Terrorismusfinanzierung zu definieren und effektiv zu bekämpfen? \\ \\ - Wie können Organisationen, die den Terrorismus finanzieren, aufgespürt und kontrolliert werden, ohne dabei die Freiheit anderer Nichtregierungsorganisationen einzuschränken? \\ \\ - Wie sollte die Weltgemeinschaft mit Staaten umgehen, die aktiv Terrororganisationen unterstützen? 


  \section{Weiterführende Links und wichtige Dokumente}
  
\href{http://www.un.org/Depts/german/uebereinkommen/ar54109.pdf}{-Resolution A/RES/54/109 der Generalversammlung zum Internationalen Übereinkommen zur Bekämpfung der Finanzierung des Terrorismus [Deutsch]} \\ \href{https://www.un.org/Depts/german/gv-60/band3/ar60288.pdf}{-Resolution A/RES/60/288 der Generalversammlung zur Weltweiten Strategie der Vereinten Nationen zur Bekämpfung des Terrorismus [Deutsch]} \\ \href{http://www.un.org/Depts/german/sr/sr_01-02/sr1373.pdf}{-Resolution des Sicherheitsrates S/RES/1373 (2001) zur Einrichtung des Ausschusses zur Bekämpfung des Terrorismus [Deutsch]} \\ \href{http://www.fatf-gafi.org/media/fatf/documents/recommendations/pdfs/FATF\%20Recommendations\%202012.pdf}{-FATF Empfehlungen zur Bekämpfung der internationalen Terrorismusfinanzierung [Englisch]} \\ \href{http://www.fatf-gafi.org/}{-Webseite der FATF [Englisch]} \\ \href{http://www.fatf-gafi.org/topics/mutualevaluations/}{-FATF Länderreports [Englisch]} \\ \href{https://www.imolin.org/}{-Webseite der IMOLIN [Englisch]} \\ \href{https://library.fes.de/pdf-files/iez/04876.pdf}{-Rolle des Ausschusses zur Bekämpfung des Terrorismus in den Vereinten Nationen und Stellung zum Sicherheitsrat [Englisch]} \\ \href{https://www.imolin.org/pdf/imolin/Role-of-hawala-and-similar-in-ml-tf-1.pdf}{-Sehr ausführlicher Bericht der FATF zur Rolle des Hawala-Systems bei der Terrorismusfinanzierung [Englisch]} \\ \href{https://www.imolin.org/pdf/overview_of_UN_conventions_2013.pdf}{-Besonders breiter Überblick über die bisherigen internationalen Maßnahmen und Konventionen gegen die globale Terrorismusfinanzierung [Englisch]} \\ \href{http://www.ag-friedensforschung.de/themen/Terrorismus/piper2.html}{-Gerhard Piper: Was ist Internationaler Terrorismus? [Deutsch]} \\ \href{http://www.un.org/depts/german/gs_sonst/09-64644_german_ctitf.pdf}{-Bericht der CTIFT zur Bekämpfung der Terrorismusfinanzierung [Deutsch]} \\ \href{https://www.transparency.de/fileadmin/Redaktion/Publikationen/2017/Scheinwerfer_76_III_2017_Organisierte_Kriminalitaet_und_Terrorismusfinanzierung.pdf}{-Bericht der Nichtregierungsorganisation Transparency International zu Organisierter Kriminalität und Terrorismusfinanzierung [Deutsch]}
 
\end{document}
